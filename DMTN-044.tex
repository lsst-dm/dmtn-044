\documentclass[DM,toc]{lsstdoc}

\input meta.tex

\usepackage{microtype}

\title[DM Software Releases]{LSST DM Software Release Considerations}
\author{John Swinbank}
\setDocRef{\lsstDocType-\lsstDocNum}
\setDocDate{\vcsdate}

\setDocAbstract{
This attempts to summarise the debate around, and suggest a path forward, for
LSST software releases. Although some recommendations are made in
\S\ref{sec:recommendations}, they are intended to serve as the basis of discussion,
rather than as a complete solution.

This material is based on discussions with several team members over a
considerable period. Errors are to be expected; apologies are extended;
corrections are welcome.
}

\begin{document}
\maketitle

\section{Consumers of Releases}

Our usual discussion over releases conflates (at least) three different
classes of consumers, which I summarize below.

\begin{enumerate}

\item{Regular LSST developers (discussed in \S\ref{sec:workflow});}

\item{Close collaborators, who are not intimately familiar with the codebase,
but require access to new features and bugfixes with minimal latency
(\S\ref{sec:collaborators});}

\item{External users, where the term ``external'' is used loosely to include
science collaborations or the Commissioning Team, who require codebase which
they can use for carrying out medium- to long-term projects without handling
regular API changes (\S\ref{sec:externals}).}

\end{enumerate}

Particularly as regards the first class of users --- LSST Developers --- the
question of releases is intimately linked with concerns over development
workflow.

\section{Developer Workflow}
\label{sec:workflow}

\subsection{Requirements \& Issues}

Our regular developer workflow is well described in the Developer
Guide\footnote{\url{https://developer.lsst.io/processes/workflow.html}}.
Fundamentally, developers perform their work on ticket branches
(\texttt{tickets/DM-NNNN}) which are reviewed and rebased onto \texttt{master}
before merging.

Note that in this model, the tip of the ticket branch immediately before
merging should be identical to the state of \texttt{master} immediately after
the merge. This means that the developer can demonstrate (e.g. using the CI
system) that merging their changes will not cause \texttt{master} to break or
regress\footnote{Merging without rebasing does not have this property: the
post-merge state of \texttt{master} will neither be the same as its pre-merge
state \textit{nor} the state of the ticket branch.}.

It is worth emphasizing that this development model means that code which hits
\texttt{master} is generally well behaved: we do not frequently have problems
with it failing to build or with bugs which render the codebase fundamentally
useless\footnote{Of course, sometimes regressions do slip in.}.

Key issues include:

\begin{itemize}

  \item{Turnaround time on reviews, in particular of complex work, can be
  long.  This can cause substantial time to pass before the work on the ticket
  branch being completed and the merge to \texttt{master}. This means:

    \begin{itemize}

      \item{Other work dependent on the new feature backs up onto a series of
      ticket branches, all based upon each other. Keeping track of which
      branches require which other branches becomes awkward.}

      \item{Since releases are made from \texttt{master}, the new
      functionality is not available in any released version of the codebase.}

    \end{itemize}

  }

  \item{\texttt{master} continually changes as other developers are merging
  their work. This means:

    \begin{itemize}

     \item{Rebasing can take substantial effort.}

      \item{Rebasing a long-running branch against a substantially changed
      \texttt{master} at the end of a long development project can be a major
      task. To minimise the effort, developers generally frequently rebase
      against the latest \texttt{master}. After rebasing, it's frequently
      necessary to rebuild all or much of the stack which (as below) can be a
      length process.}

    \end{itemize}

  }

  \item{Nominally ``high-level'' work, which superficially seems to affect
  only Python code in packages on the leaves of the tree, often also requires
  changes to lower-level packages (usually, but not always, afw). This means:

    \begin{itemize}

      \item{Rebuilding after rebasing is often a length process: it takes of
      order one hour to rebuild from a low-level package.}

    \end{itemize}

  }

  \item{Our build process is fragile: although (as above) it is rare (but not
  unknown) for the codebase to be unbuildable, it makes numerous assumptions
  about the system on which it is building, and relies on complex (and often
  poorly-understood by most developers) bespoke tooling (lsstsw, EUPS, ...).
  This means:

    \begin{itemize}

      \item{Even after successfully rebasing, developers often waste time
      debugging local build problems before they can resume work.}

    \end{itemize}

  }

\end{itemize}

\subsection{Possible Approaches}

\begin{itemize}

  \item{Enforce API/ABI stability by preventing modifications to low-level
  packages. This means:

    \begin{itemize}

      \item{Packages obviously cannot be frozen indefinitely, but rather we
      would accept changes only during certain windows.}

      \item{A release manager (\S\ref{sec:releasemanager}) may be required
      to coordinate this process.}

      \item{Work which requires changes to low-level packages could only be
      merged during dedicated windows, and would otherwise back-up.}

      \item{Any other work which happened to coincide with a window would
      encounter substantial disruption as, likely, a large number of
      interfaces would change at once.}

    \end{itemize}

  }

  \item{Require developers to develop against stable (e.g. weekly) releases,
  rather than latest \texttt{master}. This means:

    \begin{itemize}

      \item{Developers have a stable ABI and API to target.}

      \item{The ``shared stack'' available on LSST-provided developer hardware
      (\texttt{lsst-dev01}, the Verification Cluster) should always provide
      the latest weekly, so developers can develop against that with no stack
      maintenance required on their part.}

      \item{Work must still be merged to \texttt{master} (or some other
      common development branch) eventually. This means one of two things:

        \begin{enumerate}

          \item{Work is merged without rebasing. The merge is complex. The
          post-merge state of the stack is not testable until after the merge
          has taken place.}

          \item{Work is rebased from the weekly to \texttt{master} before
          merging. The rebase is potentially complex (since it involves doing
          in one step a rebase which might otherwise have been undertaken
          incrementally). The advantages vis-\`a-vis the current workflow are
          unclear.}

        \end{enumerate}

      }

    \end{itemize}

  }

  \item{Provide regularly updated \textit{binary} releases of current
  \texttt{master} (\S\ref{sec:binaryrelease}). This means:

    \begin{itemize}

      \item{Following a rebase, developers should not need to recompile any
      packages other than those on which they are directly working.}

      \item{Shared stacks on common developer hardware can be automatically
      updated with the latest release.}

      \item{Generating and installing binary releases takes time, so there is
      still a potential speed-bump to developer workflow.}

      \item{It is likely impossible to provide binaries targeting all possible
      platforms which developers may reasonably wish to use.}

    \end{itemize}

  }

  \item{Restructure stack code to better orthogonalize between packages
  (\S\ref{sec:restructure}). This means:

    \begin{itemize}

      \item{Changes to high-level packages are less likely to require changes
      to low-level packages.}

      \item{Huge low-level packages (afw) could be split into smaller
      components, so that a change would be less disruptive.}

      \item{This might help, but is not in itself a complete solution: some
      high level changes will \textit{always} require changes in low level
      packages.}

    \end{itemize}

  }

  \item{Provide better development tools, replacing or augmenting tools like
  lsstsw to make our build process more reliable, less prone to user error,
  and (where possible) faster (\S\ref{sec:devtools}). This means:

    \begin{itemize}

      \item{Developers who need to recompile after rebasing should find the
      task easier, faster and less disruptive.}

      \item{It's unclear how much benefit could really be gained here: lsstsw
      is clunky and could certainly be streamlined, but it's not a disaster.}

    \end{itemize}

  }

\end{itemize}

\section{Close Collaborators}
\label{sec:collaborators}

\subsection{Requirements \& Issues}

It is important to be able to rapidly provide collaborators, e.g. members of
the Camera Team, with versions of the stack which may resolve particular
issues or provide new features with minimal latency. However, these
individuals are unfamiliar with the stack and our particular toolchain than
regular developers. They are frustrated by awkward software and failed builds.

Collaborators may be already using a particular weekly (or other) release, and
not be in a position to deal with API or other instability. Therefore, simply
upgrading to a newer weekly may not always be convenient. Further, for the
reasons discussed in \S\ref{sec:workflow}, it may not be practical for the
work to ``simply'' be merged to \texttt{master} and hence appear in an
upcoming weekly.

However, it is accepted that access to the latest features will, at some
level, require tracking new development. It is not practical to commit to
provide arbitrary releases with an arbitrary set of bug fixes and new features
upon request.

\subsection{Possible Approaches}

\begin{itemize}

  \item{Provide better development tools (\S\ref{sec:devtools}). It seems
  unlikely that this would be able to provide a sufficiently seamless process,
  though.}

  \item{Provide easy-to-install binary distributions corresponding to
  particular branches (\S\ref{sec:binaryrelease}). This means:

    \begin{itemize}

      \item{End users could use our standard binary workflow to install a
      binary release which contains the specific features pulled onto a branch
      by their LSST contact.}

      \item{The project would make no ongoing effort to support these binary
      releases.}

      \item{When LSST developers may develop work on feature branches and
      provide it to collaborators using this method, the motivation to go
      through the effort of merging the work to \texttt{master} may be
      reduced. There is, therefore, a danger of producing a series of
      ``stale'' ticket branches which contain important functionality but
      which have never been merged. Avoiding this would require real
      discipline.}

    \end{itemize}

  }

\end{itemize}

\section{External Users}
\label{sec:externals}

\subsection{Requirements \& Issues}

External users, for a generous definition of ``external'', require a release
which provides some level of functionality and is ``supported'' in the long
term --- that is, errors discovered are fixed, and there is no expectation
that they need to upgrade to a newer version to use functionality advertised
as being part of the release.

It is not clear to what extent external users may require that new features
(as opposed to bug fixes) be added to these long-term supported releases.

The definition of ``long-term'' is vague, but we might reasonably assume the
value of six months based on current practice. Of course, the longer support
is required, the further the current development version will diverge from the
supported codebase, and the harder it will be to make fixes which apply to
both of them.

However, our current practice centres on date-based releases. In future, it
is likely that feature-driven releases will predominate as we are required to
deliver and support a particular set of functionality as required by e.g. the
Commissioning Team.

It may be necessary to maintain two (or more?) release ``trains'' in parallel,
in addition to ongoing development on \texttt{master}: for example, to support
different audiences or to provide a semi-stable branch\footnote{e.g. FreeBSD's
``STABLE'' vs ``RELEASE''.}. No concrete requirements have been expressed
here.

\subsection{Possible Approaches}

\begin{itemize}

  \item{Release procedure:

    \begin{itemize}

      \item{Releases can be made by directly selecting (and subjecting to a
      battery of tests) a particular version of the \texttt{master} branch,
      without requiring any special action from most developers. This is how
      releases have been handled to date. As per \S\ref{sec:workflow}, we have
      been relatively successful in maintaining developer discipline and
      strict practices, so that \texttt{master} is generally in a
      ``releasable'' state. This approach enables developers to continue with
      their work normally while the release happens in parallel.}

      \item{Releases could be based around a freeze of development on
      \texttt{master} to ensure that a well-tested and working version of the
      codebase is released. The details of implementing such a freeze, how
      developers might continue working while \texttt{master} is frozen, the
      branch policy, etc, are regarded as out of scope for this
      document\footnote{Although it is not current practice, historically a
      \texttt{next} branch has been used to enable ongoing development while
      \texttt{master} is frozen.}.}

    \end{itemize}

  }

  \item{Back-porting fixes:

    \begin{itemize}

      \item{While minor technical fixes will often be straightforward to
      back-port to stable releases, deciding which changes to science logic
      constitute ``bugs'' that require porting will require careful thought.}

      \item{This process might reasonably be coordinated by a dedicated
      Release manager (\S\ref{sec:releasemanager}), working in conjunction with
      the DMCCB.}

      \item{Porting substantial changes may become extremely complex as not
      just the logic but the underlying infrastructure may be quite different.
      The level of effort required is likely large.}

      \item{Fixes, even those which are known to be required by an older
      release, should \textit{always} be developed targeting \texttt{master}
      and then back-ported (unless the code being fixed has already been
      removed from \texttt{master}). No new work may be performed on release
      branches.}

    \end{itemize}

  }

\end{itemize}

\section{Technical and Resource Implications}
\label{sec:resource}

\subsection{Release Manager}
\label{sec:releasemanager}

Some of the approaches outlined above require the services of a Release
Manager. Such an individual might be required to:

\begin{itemize}

  \item{Carry out work relating to the mechanics of making releases (e.g.
  applying appropriate tags, ensuring that releases contain the requisite
  features, etc);}
  \item{Collate and compile release notes and other supporting material for
  stable releases;}
  \item{Work with the community and the DMCCB to understand which issues need
  to be back-ported to stable releases;}
  \item{Make fixes as required, including back-porting where necessary.}

\end{itemize}

Some of this work is currently being carried out by SQuaRE. Other parts of it,
in particular back-porting of bug fixes, may require development expertise
that is currently only found within other teams (e.g. Pipelines likely best
understand how to backport Pipelines code, ditto for DAX, etc). Two possible
approaches have been suggested here:

\begin{itemize}

  \item{Centralising all of this expertise in one named individual would
  streamline the process and enable easier resource loading for the
  development teams.}

  \item{Having the release manager role be simply a managerial one (which may
  rotate between project members) who can call on support from the development
  teams may reduce the concentration of skill required in one individual, at
  the expense of a more complex management and resourcing structure.}

\end{itemize}

\subsection{Code Restructuring}
\label{sec:restructure}

A significant overhaul of the structure of the codebase has been suggested on
several previous occasions. See, for example,
\href{https://jira.lsstcorp.org/browse/DM-2003}{DM-2003} or various
discussions on
Confluence\footnote{\url{https://confluence.lsstcorp.org/display/DM/Winter2015+Package+Reorganization+Planning}}\footnote{\url{https://confluence.lsstcorp.org/display/DM/Summer2015+Package+Reorganization+Planning}}.
These have generally foundered due to the large amount of work required,
the desire to address more immediately pressing concerns, and uncertainty over
long term priorities.

Broadly speaking, those concerns still apply. There is no effort currently
available within the Science Pipelines groups to spearhead a restructuring
effort, although some developer time could be devoted to supporting work
organized by another team (e.g. Architecture, SQuaRE).

\subsection{Developer Tool Upgrades}
\label{sec:devtools}

Action here might range from a relatively modest overhaul of the lsstsw script
to make it more user friendly to a root-and-branch reconsideration of the way
our software is packaged, the build tools and utilities, and the distribution
system.

The effort required would obviously vary substantially depending on how
ambitious this project was. It seems unlikely that a significant overhaul
would be worth the effort insofar as it narrowly affects pipeline developers,
but there might also be knock-on benefits for other consumers of the software
(who would find it easier to install and better integrated with their typical
environment, e.g. pip, conda, etc.).

There is no effort currently available within the Science Pipelines groups to
spearhead a restructuring effort, although some developer time could be
devoted to supporting work organized by another team (e.g. Architecture,
SQuaRE).

\subsection{Binary Release Procedures}
\label{sec:binaryrelease}

The SQuaRE team is developing a system which makes it possible to produce
binary releases from Jenkins runs. Specifically:

\begin{itemize}

  \item{They directly anticipate being able to generate a binary release from
  ticket branches upon request.}

  \item{They are open to the possibility of generating binaries from all
  builds of \texttt{master} (or, presumably, based on some more sophisticated
  selection algorithm).}

\end{itemize}

Note that a binary release currently consumes around 5\,GB of storage.
Long-term archiving of binaries corresponding to \textit{every} build of
\texttt{master} is likely impractical given the storage considerations.

It would take some effort for SQuaRE to develop a system capable of generating
binaries for all \texttt{master} builds, including time devoted to developing
sensible resource management systems, etc. However, this is likely possible
upon request.

Note that the established variety of platforms upon which team members and
their collaborators are working, together with issues such as C++ ABI
compatibility \footnote{e.g.
\url{https://wiki.gentoo.org/wiki/Upgrading\_GCC\#The\_special\_case\_C.2B.2B11\_.28and\_C.2B.2B14.29}}
mean that it is likely impossible to guarantee to provide binaries compatible
with every system.

\section{Recommendations}
\label{sec:recommendations}

Given the above considerations, the personal recommendations of the author of
this document are as follows. These are not normative and should not be taken
as expressing DM subsystem policy.

\subsection{Developers}

\begin{itemize}

  \item{Provide regularly updated binary releases of current \texttt{master},
  automatically installing them onto shared developer architecture.}

  \item{Encourage developers to develop against weekly releases wherever
  possible, making use of binaries to help with rebasing.}

  \item{Schedule minor package reorganization and development tool improvement
  work when possible and when they align with other goals, but do not
  anticipate a major overhaul in this area.}

\end{itemize}

\subsection{Close Collaborators}

\begin{itemize}

  \item{Make use of the ability to generate binary releases from ticket
  branches to provide collaborators with access to new features and bug fixes
  with low latency and minimal effort on their part.}

  \item{Rely on disciplined management to ensure that those features \& fixes
  are also merged to \texttt{master} in a timely fashion.}

\end{itemize}

\subsection{Long Term Support}

\begin{itemize}

  \item{Appoint a Release Manager to coordinate long-term supported releases
  in conjunction with the DMCCB. This individual would (presumably) be part
  of the SQuaRE team.}

  \item{Make releases by tagging \texttt{master}, then having the Release
  Manager maintain the branch by back-porting essential fixes.}

  \item{Where necessary, multiple stable branches may be maintained.
  Developers will continue working on \texttt{master} with the RM and DMCCB
  back-porting work.}

  \item{Consider switching from a time-based to a feature-based release
  schedule.}

\end{itemize}

\section{Conclusions}
\label{sec:conclusion}

At the
\href{https://confluence.lsstcorp.org/display/DM/DM+Leadership+Team+Meeting+2017-05-09+to+11%2C+Face-to-Face}{May
2017 DMLT Face-to-Face Meeting}, held at the University of Washington, the DM
Leadership Team discussed the status of binary releases and the contents of
this document. The following conclusions were reached:

\begin{itemize}

  \item{The binary distribution system developed by SQuaRE
  (\S\ref{sec:binaryrelease}) and described at the meeting will meet the
  short term needs of DM developers and close collaborators.}

  \item{The DM-SST identifies a possible need to reconsider
  \href{https://conda.io}{Conda}-based binary releases to meet the future
  needs and expectations of the wider community. No action will be taken on
  this in the short term.}

  \item{Long term stable release plans will be finalized when a Release Manager has
  joined the team. Recruitment efforts for this position are currently getting
  underway in Tucson.}

  \item{It is likely that future, long-term supported releases will be
  feature, rather than date, driven. A detailed policy will be put in place
  when the Release Manager joins the project.}

\end{itemize}

As the new Release Manager comes onboard and develops the relevant policies,
they will be recorded in \citeds{LDM-294} and, if necessary, further documents
cited from there.

\section{References\label{sect:references}}
\renewcommand{\refname}{}
\bibliography{lsst}

\end{document}
